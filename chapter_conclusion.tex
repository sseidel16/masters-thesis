\chapter{Conclusion} \label{chptr:conclusion}

This thesis began by introducing locating arrays and providing motivation for their use in the screening process.
It showed how covering arrays are useful in determining the presence of a significant $t$-way interaction, but that coverage is not enough.
Screening requires that significant $t$-way interactions be distinguished from each other so that the relevant system factors may be identified.
Locating arrays are therefore a natural solution for the screening process.
Next, this thesis discussed construction of locating arrays, analysis strategies, and robustness.

However, throughout this thesis, we only examined locating arrays with $d = 1$ because of a "heavy-hitters" assumption.
However, one cannot be sure that the effects of all $t$-way interactions always follow this pattern.
We therefore leave it for  future work to examine locating arrays with $d \geq 1$ to cope with terms that do not conform to the "heavy-hitters" pattern.

We also assumed in this thesis that interactions of strength greater than 2 are not significant.
But there are certainly situations where interactions of strength 3 or higher significantly affect the response.
In this case, the work in this thesis may seek to estimate these effects with main effects and 2-way interactions likely leading to false screening results.
We leave it for future work to examine locating arrays with interactions of strength greater than 2.

Following the introduction of locating arrays, this thesis discussed construction approaches for locating arrays.
Building on related work, this work showed that locating arrays can be created using the Lov{\'a}sz Local Lemma with Moser-Tardos resampling, and that the construction process can be accelerated by using a scoring procedure for the array.
Furthermore, Moser-Tardos resampling can be used to construct locating arrays that have additional constraints including separation $\delta$.

Next, a new analysis approach for locating arrays was presented in Algorithm \ref{alg:bfs_analysis}.
We build on iterative SIS and OMP in related work to create an algorithm that is both useful and easy to implement.
However, several aspects of our analysis algorithm should be investigated in future work.
Our analysis approach in Algorithm \ref{alg:bfs_analysis} always keeps the same number of models, $\mathit{nModels}$, in its queues, and always explores the same number of alternatives, $\mathit{nNewModels}$.
But there may be certain iterations where it is better to explore more options because of high uncertainty and other iterations where this is not required because of high certainty.
Ultimately, while there may be benefits to keeping these as constant parameters to the algorithm, one might instead want to vary these parameters dynamically.
Furthermore, one might also want the parameter $nTerms$ to be chosen automatically to correct possible overfitting issues with analysis.
This might be done by setting a threshold for $R^2$, or the analysis might be updated to stop adding terms when $R^2$ does not improve significantly.
Yet another possibility is to track the occurrence counts as more terms are added and stop adding terms based on the drop-off in the factor rankings.
This work does not explore these possibilities but leaves them for future work.

One important aspect of locating arrays affecting analysis is their imbalance.
Although all main effects and interactions (up to strength $t$) are covered in a $(\overline{d},\overline{t})$-locating array, some terms are covered much more than others.
Analysis strategies for locating arrays should not be biased towards terms that are covered more often than others.
However, OMP uses a dot product that treats all tests equally and ignores the imbalance that is inherent to locating arrays.
Suppose a locating array $A$ contains hundreds of tests but a particular interaction $T_1$ is covered in exactly one test.
If $T_1$ is relevant to the system under test, then it is likely that OMP, which treats all tests equally, will be biased towards terms that affect the response in several tests and $T_1$ will experience discrimination.
There also exist limitations with using $R^2$ to measure the goodness of models.
All tests are again treated equally when calculating $R^2$, and thus it is likely that the effects of $T_1$ are ignored with little change to $R^2$.

One possible solution might be to weight the tests of a locating array based on the percentage of coverage for a particular interaction that each test accounts for.
Yet each test involves many interactions that are each covered differently leading to more challenges.
To complicate matters even further, system noise affects all tests equally, and placing a higher weight on a particular test will exacerbate noise in that test.
In \cite{AldacoCS15}, strategies are employed to specifically cope with imbalance in the locating array.
Interestingly, this thesis produces nearly identical results for the same screening experiment in \cite{AldacoCS15} even though we do not consider imbalance.
It may be interesting to investigate why these approaches that are quite different produce the same results, and if there are other experiments where the approaches might produce different results.
We leave this difficult issue of balance and unfair bias of tests to future work.

Algorithm \ref{alg:bfs_analysis} analyzes the locating array response and generates a list of $nModels$ best models.
But screening must indicate what factors are significant to include in experimentation.
This thesis determines the significant factors by counting the occurrences of each factor in the list of best models generated by the analysis algorithm.
Although counting occurrences of factors might be simple, this approach may not be the best way to determine significance.
For example, a factor appearing in a main effect with a large coefficient is likely more significant than a factor appearing in a main effect with a small coefficient.
A factor is also likely more important when it appears in a main effect than when it appears in an interaction.
And the significance of a factor likely also depends on the goodness of the model it appears in (the $R^2$ of the model).
However, this thesis ignores coefficients and model goodness when counting factor occurrences.
A factor appearing in an interaction is also counted exactly the same as when it appears in a main effect.
However, we leave it for future work to examine and resolve these issues.

Finally, this thesis discussed robustness of our screening approach using locating arrays.
We discussed negative effects that may occur when the "heavy-hitters" assumption is not satisfied.
Scenarios were provided with systems satisfying the "heavy-hitters" assumption to varying degrees.
Furthermore, artificial noise was introduced in several scenarios and the results were compared to the same scenarios but without noise.
We showed how separation can help fight against the effects of noise.
Scenarios were also given showing systems that included a large number of terms, and these terms obscuring each other's effects.
